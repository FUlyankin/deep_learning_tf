%!TEX TS-program = xelatex
\documentclass[12pt, a4paper, oneside]{article}

\input{preamble.tex}

\begin{document}

\section*{Quiz 6: NLP и автокодировщики}

\epigraph{ -- Не ходи туда, там тебя ждут неприятности. \\ -- Ну как же не ходить они же ждут...}{\textit{Котенок по имени Гав (1976)}}

Решите все задания. Все ответы должны быть обоснованы. Решения должны быть прописаны для каждого пункта. Рисунки должны быть чёткими и понятными. Все линии должны быть подписаны. Списывание карается обнулением работы. \indef{При решении работы можно пользоваться чем угодно.} Удачи! 

\vspace{-0.5cm}
\subsection*{[5] Задание 1}
\vspace{-0.5cm}
Мы пытаемся сжать картинку $v_i$ с помощью метода главных компонент. Давайте запишем эту задачу в виде автокодировщика. 
\begin{itemize} 
    \item Запишите формулы для энкодера и декодера. Из каких пространств в какие они бьют как функции? 
    \item Выпишите функцию потерь, которая будет использоваться при решении задачи для обучения. 
\end{itemize} 

\vspace{-0.5cm}
\subsection*{[3] Задание 2}
\vspace{-0.5cm}
Объясните что такое негативное сэмплирование (negative sampling) и для чего оно нужно при обучении w2v.

\vspace{-0.5cm}
\subsection*{[2] Задание 3}
\vspace{-0.5cm}
Томаш обучает w2v для английского языка на корпусе новостей с помощью метода skip-gram. Сколько параметров ему надо будет оценить при обучении модели, если он собирается оставить в словаре $100 000$ самых частотных токенов? 

\vspace{-0.5cm}
\subsection*{[1] Задание 4}
\vspace{-0.5cm}
Мы хотим, чтобы эмбеддинги рассуждали также, как это делают люди. Давайте, наоборот, попробуем рассуждать как нейросети, обучившиеся на каком-то корпусе текстов и словившие странные артефакты. Предположите, как решаются следующие уравнения и кратко поясните почему вы так думаете.

\begin{enumerate} 
    \item  ночь - темнота + свет = ? 
    \item  сосиска - маленькая + большая = ? 
    \item  python - язык + алкоголь = ? 
    \item  цб - резервы + проблемы = ?
\end{enumerate} 


\end{document}