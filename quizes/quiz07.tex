%!TEX TS-program = xelatex
\documentclass[12pt, a4paper, oneside]{article}

\input{preamble.tex}

\begin{document}

\section*{Quiz 7: Внимание и трансформеры}

\epigraph{All you need is love \\ All you need is love, \\ All you need is love, love\\ Love is all you need.}{\textit{The Beatles, (1967)}}

Решите все задания. Все ответы должны быть обоснованы. Решения должны быть прописаны для каждого пункта. Рисунки должны быть чёткими и понятными. Все линии должны быть подписаны. Списывание карается обнулением работы. \indef{При решении работы можно пользоваться чем угодно.} Удачи! 

\vspace{-0.5cm}
\subsection*{[6] Задание 1}
\vspace{-0.5cm}
Кратко, но ёмко дайте ответы на следующие вопросы:
\begin{itemize} 
    \item  Почему трансформеры на длинных последовательностях работают лучше, чем рекуррентные нейронные сети? 
    
    \item  Объясните, как в рамках трансформера можно обучить нейросеть учитывать порядок токенов в последовательности
    
    \item  Опишите, что из себя представляет BERT. На какие задачи можно его обучать? 
\end{itemize} 


\vspace{-0.5cm}
\subsection*{[2] Задание 2}
\vspace{-0.5cm}

Предположим, что мы хотим обучить архитектуру, которая будет получать на входе аудиофайл, а на выходе будет превращать его в текст. Будем считать, что аудиодорожка хранится на компьютере в виде последовательности из чисел. Какую бы архитектуру вы использовали для решения этой задачи? Кратко опишите её части. 


\vspace{-0.5cm}
\subsection*{[2] Задание 3}
\vspace{-0.5cm}

Мы хотим использовать механизм внимания для решения задачи генерации текста по картинке. Предложите способ добавить механизм внимания для изображения в свёрточную архитектуру. 

\vspace{-0.5cm}
\subsection*{[1] Задание 4}
\vspace{-0.5cm}
Объясните, почему в эпиграфе к работе находится строки из песни группы The Beatles. 


\end{document}