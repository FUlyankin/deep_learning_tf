%!TEX TS-program = xelatex
\documentclass[12pt, a4paper, oneside]{article}

\input{preamble.tex}

\begin{document}

\section*{Quiz 4: свёрточные нейронные сети}

\epigraph{Зло — это зло, Стрегобор, — серьёзно сказал ведьмак, вставая. — Меньшее, большее, среднее — всё едино, пропорции условны, а границы размыты. Я не святой отшельник, не только одно добро творил в жизни. Но если приходится выбирать между одним злом и другим, я предпочитаю не выбирать вообще.}{\textit{Ведьмак, рассказ «Меньшее зло»}}


Решите все задания. Все ответы должны быть обоснованы. Решения должны быть прописаны для каждого пункта. Рисунки должны быть чёткими и понятными. Все линии должны быть подписаны. Списывание карается обнулением работы. \indef{При решении работы можно пользоваться чем угодно.} Удачи! 

\vspace{-0.5cm}
\subsection*{[2] Задание 1}
\vspace{-0.5cm}

Алекс не понимает, почему применять полносвязные нейронные сети для работы с изображениями --- плохая идея. Кратко объясните ему это.

\vspace{-0.5cm}
\subsection*{[4] Задание 2}
\vspace{-0.5cm}

Архитектура AlexNet работает с картинками размера $227 \times 227 \times 3$.  Первый свёрточный слой содержит в себе $96$ свёрток размера $11 \times 11$ и параметр сдвига (stride) равный $4$. Дополнение нулями (padding) не используется. 

\begin{enumerate} 
    \item[а)]  Какого размера будет картинка, когда она пройдёт сквозь этот слой?  
    
    % (227 - 11)/(4 + 1) = 55
    % W’ = (W - F + 2P) / S + 1
    % $55 \times 55 \times 96$.
    
    \item[б)] Какое число параметров надо оценить?
    % (11*11*3)*96 = 35k 
\end{enumerate}


После первого слоя картинка попадает в слой пулинга размера $2 \times 2$. 
\begin{enumerate} 
    \item[в)]  Какого размера будет картинка, когда она пройдёт сквозь этот слой?  
    
    % (55 - 3)/2 + 1 = 27
    % Получается, что $27 \times 27 \times 96$.
    \item[г)] Какое число параметров надо оценить?
    % ноль 
\end{enumerate}

\vspace{-0.5cm}
\subsection*{[4] Задание 3}
\vspace{-0.5cm}

Алекс, Илья и Джеффри обучают свёрточные нейронные сети. Объясните, чей подход правильный и почему.

\begin{itemize} 
\item Алекс использует свёртку $3 \times 3$ с $ReLU,$ а затем использует max-pooling. 
\item Илья использует свёртку $3 \times 3$, затем делает max-pooling, а после применяет $ReLU.$
\item Джеффри делает свёртку $3 \times 3$, а после max-pooling без функции активации. 
\end{itemize} 

\vspace{-0.5cm}
\subsection*{[1] Задание 4}
\vspace{-0.5cm}

Объясните, почему в третьей задаче используются именно такие имена персонажей.

\end{document}